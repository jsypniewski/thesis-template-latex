% put environments that should be ignored by texcount here, e.g., here lstlisting for code

%TC:envir lstlisting [] ignore

%for reference to this section
\section{Introduction}
\label{section:Introduction}
Explanation and justification of the topic (argumentation phase – introduction): In this phase of the thesis, it is necessary to explain the reasons why the topic of the thesis or the question is relevant and why the topic should be addressed. The motivation for the topic therefore stands in the foreground, though this refers not to personal preferences, but rather the topical significance of the question in the context of research and innovation.

\section{Theoretical Grounding \& Related Work}
Presentation of current knowledge on the topic (literature phase – basic knowledge): An overview of the topic and concepts and a description of the current state of research (e.g. possible solutions, technologies, theory) should be presented here based on sources from the scientific-technical literature. Special attention should be paid to the correct citation of sources. This should help other readers of the paper to understand the subsequent parts of the paper. Contents of electronic sources (websites) are acceptable, but must be saved.

Discussion of possible methods and solutions (feasibility phase – analysis): This section begins with a comparison of methods of scientific approach to the topic that are possible in principle, and an assessment and discussion of the chosen method’s likelihood of success. Based on the existing literature, here it is necessary to argue which gap this work fills and which questions it addresses. This should be followed by an explanation of different approaches to the topic based on specific variants, in light of the results of the literature phase on the one hand, and with regard to innovative solutions for different fields of application on the other.

\section{Research Question(s) \& Approach}
Proceeding on the basis of the previous chapters, the specific research question(s) to be addressed in the thesis should be described here (referring to the content of the previous chapter). Furthermore, the specific approach, i.e., the methodology for answering the chosen question(s), should be described in detail, including possible ethical aspects (cf. Research Ethics).

\section{Results}
Presentation of solution and results (realisation phase): The independently developed solution or partial solution to a problem or the research question is described in detail.

\section{Discussion}
Discussion of results (justification phase): The critical discussion of the technical or analytical work steps is presented here and the relevance to innovation is established. Comparisons with other results on similar questions in the literature are entirely appropriate (reference to questions and knowledge gaps in existing literature on the topic identified in the “Related Work” chapter). Arguments may also be presented here concerning social relevance and responsibility with regard to the research question, or it may be placed it in the context of an overall social discourse.

\section{Reflection \& Outlook}
The thesis is rounded off with a recapitulation of the main findings and a look ahead to possible further development of the techniques and/or methods chosen. Special attention is paid here to the possibility of innovation in a specific field of application or the further scientific questions that arise from the work. 

% \section{Formatierung}
% \label{section:Formatting}

% Text mit beliebigen Sonderzeichen in UTF-8 ohne BOM \ldots
% ,
% \textbf{hervorgehobener Text},
% \texttt{void}\footnote{Fußnote 1},
% mathematische Formel im Text $\sum_{i=0}^n i^2$
% \ldots

% Referenz auf Unterabschnitt \ref{subsection:Coding} der Arbeit, automatisch richtig nummeriert.

% \textcite[]{Mulloni:2010} für einen einen Literaturverweis im laufenden Text.

% Literaturverweise sind essentiell für eine wissenschafliche Arbeit. \autocite[]{McConnell:2004:CCS:1096143}.

% Achtung: nur zitierte Literatur wird im Literaturverzeichnis
% angeführt.\footnote{Fußnote 2}


% Wir verwenden \LaTeX\footnote{ \url{http://en.wikibooks.org/wiki/LaTeX}} -- und das
% ist keine Quelle, sondern blos eine URL.

% \subsection{Figures machen was sie wollen}

% % h = try to place the figure Here
% % t = try to place the figure at the Top of a page
% % p = try to place this figure along with others on a separate Page
% % Note that LaTeX has a sophisticated ranking algorithm to place figures.
% % It is not always easy to accept LaTeX's placing but it is harder doing it
% % manually. Just let it go ;-)
% \begin{figure}[!ht]
% 	\centering
% 	\subfloat[Das Julia Fraktal]{
% 		\includegraphics[width=0.75\linewidth]{images/Julia-Fractal.png}
% 		%for reference of this subfigure only
% 		\label{subfigure:Julia-Fractal}
% 	}
% 	\qquad
% 	\subfloat[Noise für Tinteneffekte]{
% 		\includegraphics[width=0.75\linewidth]{images/Perlin-Coherent.png}
% 		%for reference of this subfigure only
% 		\label{subfigure:Perlin-Coherent}
% 	}
% 	\caption[
% 		Verschiedene Pixelgraphiken\newline
% 		% source url given in the table of figures
% 		\small\texttt{https://mediacube.at/wiki/}
% 	]{
% 		Verschiedene Pixelgraphiken
% 	}
% 	%for reference to all subfigures
% 	\label{figure:PixelImages}
% \end{figure}

% Unterstützte Pixelgraphikformate: PNG, JPEG, PDF.
% Angabe von height oder width meist wichtig.

% Referenz auf Abbildung \ref{figure:PixelImages} mit allen Teilbildern.
% Referenz auf Unterabbildung \ref{subfigure:Julia-Fractal}.

% %figure* stretches figure over both columns
% \begin{figure*}[t]
% 	\centering
% 	\includegraphics[width=0.9\textwidth]{images/KappaGamma.pdf}
% 	\caption{
% 		Vektorgraphik mit \LaTeX\ Beschriftung ($\kappa$, $\gamma$)
% 	}
% 	%for reference to this figure
% 	\label{figure:KappaGammaTau}
% \end{figure*}

% Referenz auf Abbildung \ref{figure:KappaGammaTau}.

% Bei Vektorgraphik mit \LaTeX\ Beschriftung keine Skalierung mit width oder height verwenden.

% Vektorgraphik mit \LaTeX\ Beschriftung kann etwa mit \texttt{ipe} erstellt werden.

% Unterstütztes Vektorgraphikformat: PDF. EPS muss konvertiert werden.


% \subsection{Unterabschnitt 2}
% %for references to this subsection
% \label{subsection:Coding}

% \begin{lstlisting}[
% 	label=listing:Main, %for reference to this listing
% 	float=h,
% 	caption=main.cpp,
% 	firstnumber=10
% ]
% int main(void) {
% 	while (true) {
% 	}
% 	return 0;
% }
% \end{lstlisting}

% Wie man in Listing \ref{listing:Main} in Zeile 10 sieht, kann man die Zeilennummern im Listing absichtlich setzen, hier z.B. auf 10. In Listing \ref{listing:closure} wurde davon nicht Gebrauch gemacht. In diesem Fall beginnt die Nummerierung bei 1.

% \begin{lstlisting}[
%     label=listing:closure,
% 	float=h,
% 	caption=Closure in Javascript,
% 	language=JavaScript
% ]
% function foo(x,y) {
%     let i = x;
%     return function(a) {
%         return i * 2;
%     }
% }
% \end{lstlisting}


% \subsubsection{Unterunterabschnitt i}

% Wörtliches Zitat:
% %select proper language if not in German
% \selectlanguage{english}
% \begin{quote}
% ``Erwin Unruh discovered that templates can be used to compute
% something at compile time. [...] The intriguing part of this exercise, however, was that the production of the prime numbers was performed by the compiler during the compilation process and not at run time.''

% \autocite[305]{Bosch2014}
% \end{quote}
% %select German again or the language that you were using before (note ngerman stands for New German)
% %\selectlanguage{ngerman}
% \selectthesislanguage


% \subsection{Unterabschnitt b}

% \begin{enumerate}
% 	\item Punkt 1
% 	\begin{enumerate}
% 		\item Unterpunkt 1
% 		\item Unterpunkt 2
% 	\end{enumerate}
% 	\item Punkt 2
% \end{enumerate}

% \begin{itemize}
% 	\item Punkt 1
% 	\begin{itemize}
% 		\item Unterpunkt 1
% 		\item Unterpunkt 2
% 	\end{itemize}
% 	\item Punkt 2
% \end{itemize}


% \subsection{Unterabschnitt c}

% \begin{table}[ht]
% 	\centering
% 	\begin{tabular}{r|rrr}
% 		    & $i$ & $j$ & $k$ \\ \hline
% 		$i$ &$-1$ & $k$ &$-j$ \\
% 		$j$ &$-k$ &$-1$ & $i$ \\
% 		$k$ & $j$ &$-i$ &$-1$
% 	\end{tabular}
% 	\caption{
% 		Multiplikationstabelle für Quaternionen
% 	}
% 	\label{table:Quaternions}
% \end{table}

% Referenz auf Tabelle \ref{table:Quaternions}.

% \section{Abschnitt 2}
% \label{section:MathematicalStuff}

% Sei $f(x)$ eine stetige Funktion, so ist die \textbf{Fourier Transformierte}
% $F(\omega)$ wie folgt definiert:
% \begin{equation}
% \label{equation:FourierDefinition}
% 	F(\omega) = \int_{-\infty}^{\infty} f(x) e^{-i\omega t} dt
% \end{equation}

% Referenz auf mathematische Gleichung (\ref{equation:FourierDefinition}).

% Unnummerierte Gleichung:
% \begin{equation*}
% 	e^{i\varphi} = \cos\varphi + i \sin\varphi
% \end{equation*}
% %you may also use \[ \] instead of \begin{equation*} and \end{equation*}

% Gleichungssystem:
% \begin{eqnarray}
% 	g(x) = f(x - x_0) & \Leftrightarrow &
% 		G(\omega) = F(\omega) e^{-i\omega x_0} \\
% 	g(x) = f(x) e^{i\omega_0 x} & \Leftrightarrow &
% 		G(\omega) = F(\omega - \omega_0)
% \end{eqnarray}
